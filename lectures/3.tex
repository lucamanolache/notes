\lecture{1}{Jan 23}{Induction}

\section{Induction}

Induction is used to prove statements of the form:
\begin{equation*}
  (\forall k \in \mathbb{N})(P(k))
\end{equation*}

The basic form for mathematical induction is:
\begin{itemize}
  \item Prove $P(0)$ (the base case)
  \item Prove $P(k) \implies P(k+1)$
  \begin{itemize}
    \item Assume $P(k)$ (induction hypothesis)
    \item Prove $P(k+1)$ (induction step)
  \end{itemize}
\end{itemize}

\begin{note}
  We get to use $P(k)$ to prove $P(k+1)$. We only prove local things for $k$ and
  $k+1$.
\end{note}

\section{Two Color Theorem}

\begin{thm}[Four color theorem]
  Any map in a plane can be colored using four-colors in such a way that
  regions sharing a common boundary (other than a single point) do not share
  the same color
\end{thm}

This theorem is way too hard to prove, so instead we focus on a simplified
theorem called the two color theorem.

\begin{thm}[Two color theorem]
  Any map formed by dividing the plane into regions by drawing straight lines
  can be properly colored with two colors.
\end{thm}


\begin{proof}
  Start with one line dividing the plane. This can be clearly divided in two
  colors.

  Add a line. Each added line will get the previous one and then fixed
  conflicts by switching on one side.
\end{proof}
