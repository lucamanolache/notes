\lecture{2}{Jan 18}{Proofs}

\section{Background and Notation}
\begin{definition}[Integers]
  Integers are closed under addition.

  \[
    a,b \in \mathbb{Z} \implies a + b \in \mathbb{Z}
  \]
\end{definition}
\begin{definition}[Divisable]
  $a|b$ is read as "$a$ divides $b$".

  Formally, $a|b \iff \exists q\in\mathbb{Z} \text{ where } b=aq$.
\end{definition}
\section{Direct Proof}
To prove $P \implies Q$ you assume $P$ is true and use that to prove $Q$.
\section{Proof by Contraposition}
Remember \ref{contra}, $(P \implies Q) \equiv (\lnot Q \implies \lnot P)$.
Therefore, if we assume $\lnot Q$ and can prove that it implies $\lnot P$, then
we have proved the original goal of $P \implies Q$.
\section{Proof by Contradiction}
Show that if we assume $\lnot P$, then we use forward reasoning (direct proof)
to show $\lnot P \implies R$ and to show $\lnot P \implies \lnot R$. This is
often found when trying to show a simple property should always not hold.
\section{Proof by Cases}
Prove all possible cases for something are true, therefore the thing you are
proving must be true.

