\lecture{1}{Jan 16}{Propositional Logic}

\section{Propositions}
\begin{definition}[Propositions]
  Statements that are true or false.
\end{definition}

Some example propositions are $\sqrt{2}$ is irrational, $2+2=4$, and $2+2=3$.
Some non examples are $4+5$ and $x+x$.
\section{Propositional Forms}
Propositions can be put together to make another proposition. Some examples of
this are conjunction ($P \land Q$), disjunction ($P \lor Q$), and negation ($\lnot P$).

\begin{example}
  Sample propositional forms:
\begin{enumerate}
  \item $\lnot (2+2=4)$ is false
  \item $(2+2=3)\land (2+2=4)$ is false
\end{enumerate}
\end{example}

\section{Implication}
\begin{definition}[Implication]
  $P \implies Q$ can be read as "If $P$, then $Q$."

  Implications are only false if $P$ is true and $Q$ is false. When $P$ is false,
  the implication will be true, however this is a meaningless statement.
  
  Additionally, an implication is equivelent to

  \begin{equation}
    P \implies Q \equiv \lnot P \lor Q
  \end{equation}
  \begin{table}[htpb]
    \centering
    \caption{Truth table for $P \implies Q$}
    \label{tab:label}
    \begin{tabular}{|c|c|c|}
      $P$ & $Q$ & $P \implies Q$ \\
      T & T & T \\
      T & F & F \\
      F & T & T \\
      F & F & T
    \end{tabular}
  \end{table}
\end{definition}

\begin{example}
  If you stand in the rain, then you'll get wet.

  $P=$ "you stand in the rain"

  $Q=$ "you get wet"

  $P \implies Q$
\end{example}

\begin{note}
  If $P \implies Q$, that does not say anything about the opposite of $Q \implies P$.
\end{note}

\begin{definition}[Contrapositive]
  \label{contra}
The contrapositive of $P \implies Q$ is $\lnot Q \implies \lnot P$.
They are logically equivelent to the implication.
\end{definition}

\begin{definition}[Converse]
The converse of $P \implies Q$ is $Q \implies P$.
They are NOT logically equivelent to the implication.
\end{definition}

\begin{definition}[If and only if]
  If $P\implies Q \land Q \implies P$, then P if and only if (iff) Q or $P \iff Q$.
\end{definition}

\section{Truth Tables}
Truth tables can simplify doing calculations with propositions.

Truth tables are also able to be used to prove logical equivalence.
If two statements have the same truth tables, they are logically equivelent.

\section{Quantifiers}
None of the below are propositions as they have a free variable.
\begin{itemize}
  \item $\sum_{i=1}^{n} i = \frac{n(n+1)}{2}$
  \item $x > 2$
  \item $n$ is even and the sum of two primes
\end{itemize}
These are all called predicates.
They are similar to a function which returns true or false.

To turn them into a predicate we need a quantifier.

\begin{definition}[Universe]
  A universe is the type of a variable. Some examples are:
  \begin{itemize}
    \item $N = \{0, 1, 2, \ldots\}$
    \item $N^+ = \{1, 2, 3, \ldots\}$
  \end{itemize}
\end{definition}

\begin{definition}[There Exists]
  $(\exists x \in S)(P(x))$ means $P(x)$ is true for some $x$ in $S$.

  Example: $(\exists x \in N) (x = x^2)$
  True because $0\times 0 = 0$ and is equivelent to $(0=0) \land (1=1) \land(2=4)\land\ldots$.
\end{definition}

\begin{definition}[For all]
  $(\forall x \in S)(P(x))$ means $P(x)$ is true for all $x$ in $S$.

  Example: $(\forall x \in N) (x+1>x)$
  True because adding one to a number will always be larger.
\end{definition}

\begin{note}
  Quantifiers are not communitive.
\end{note}

\section{DeMorgan's Law for Quantifiers}

When negating a quantifier, you negate the inside and flip the quantifier.
This means
\begin{equation}
  \lnot (\forall x \in S)(P(x)) \equiv (\exists x \in S)(\lnot P(x))
\end{equation}
and
\begin{equation}
  \lnot (\exists x \in S)(P(x)) \equiv (\forall x \in S)(\lnot P(x))
\end{equation}
